\documentclass[a4paper, 10pt]{article}
\usepackage[unicode=true, colorlinks=true, linkcolor=blue, urlcolor=blue]{hyperref}
\usepackage[T2A]{fontenc}
\usepackage[utf8]{inputenc}
\usepackage[russian]{babel}
%\usepackage{indentfirst}
\usepackage{amssymb}
\usepackage{enumitem}
\usepackage{hyperref}
\usepackage{geometry}
\usepackage{mathtools}
\usepackage{setspace}
\usepackage{tikz}
\usepackage{textcomp}
\usepackage{ stmaryrd }
\usepackage{ulem}
\usepackage{ dsfont }
\usepackage{tikzsymbols}
\usepackage{graphicx}
\usepackage{graphicx}
\graphicspath{{img/}}
\DeclareGraphicsExtensions{.pdf,.png,.jpg}

% Эта штука делает так, что у первого абзаца есть отступ. Почему-то по дефолту в техе все отступы начинаются со второго. Можешь закомментировать и посмотреть, что из этого получится
\usepackage{indentfirst}
\newcommand{\FF}{\mathcal{F}}
\newcommand{\RR}{\mathcal{R}}
\newcommand{\CC}{\mathcal{C}}
\newcommand{\MM}{\mathcal{M}}
\newcommand{\II}{\mathcal{I}}
\newcommand{\NN}{\mathcal{N}}

\geometry{a4paper,top=2cm,bottom=2cm,left=2cm,right=2cm}
\usepackage{hwspecial}

%$$\includegraphics{}$$

\title{Коллоквиум по дискретной математике 2}
\date{}
\begin{document}

\maketitle

\tableofcontents

\newpage

\section{Логика и машины Тьюринга}

\subsection{Структуры и сигнатуры. Нормальные структуры. Изоморфизм структур.}

Структура -- кортеж множеств $(M, \FF, \RR, \CC)$, где 
\begin{enumerate}
    \item $M$ -- непустое множество, \textit{носитель структуры}
    \item $\FF$ -- множество функций вида $f\colon M^n\to M$
    \item $\RR$ -- множество кортежей из $M$
    \item $\CC$ -- подмножество $M$
\end{enumerate}

Сигнатура -- кортеж попарно непересекающихся множеств $(Fnc,Prd,Cnst)$, где $Fnc$ -- множество функциональных символов, $Prd$ -- непустое множество предикатных символов и $Cnst$ -- множество константных символов. (просто набор символов)

*$\sigma$-структура (или интерпретация сигнатуры $\sigma$) -- это формально кортеж $\MM = (M,\FF,\RR,\CC,\II)$, где $\II(Fnc)=\FF,~\II(Prd)=\RR$ и $\II(Cnst)=\CC$. Вводим обозначения: $\II(Fnc)=f^\MM$, $\II(Prd)=R^\MM$ и $\II(Cnst)=c^\MM$. Для задания $\sigma$-структуры достаточно только $M$ и $\II$.

Нормальная структура -- содержащая двувалентный предикатный символ ``=`` $:=~\{(a, a) \in M^2~|~a \in M\}$, где $M$ -- носитель структуры.

% **Инъекция -- отображение такое, что из $f(x_1)=f(x_2)$ следует $x_1=x_2$.

Изоморфизм структур: интепретации $\MM$ и $\NN$ сигнатуры $\sigma$ с носителями $M$ и $N$ соответственно изоморфны если существует биекция $\eta \colon M \to N$ для которой выполняются следующие свойства:
\begin{enumerate}
    \item $\eta(f^\MM(a_1,\ldots,a_n))=f^\NN(\eta(a_1),\ldots,\eta(a_n))$
    \item $(a_1,\ldots,a_n)\in R^\MM \iff (\eta(a_1),\ldots,\eta(a_n))\in R^\NN$
    \item $\eta(c^\MM) = c^\NN$
\end{enumerate}

\subsection{Формулы первого порядка данной сигнатуры. Параметры (свободные переменные)
формулы. Предложения.}

Формулы первого порядка -- это выражения в логике первого порядка (предикатной логике), построенные по правилам синтаксиса, установленным для данной сигнатуры.

Формулы первого порядка строятся из термов и предикатов, используя логические связки и кванторы. Основные элементы синтаксиса формул первого порядка:
\begin{enumerate}
    \item Термы: переменные, константы и функции, примененные к термам.
    \item Атомарные формулы: предикаты, примененные к термам.
    \item Сложные формулы: атомарные формулы, соединенные логическими операциями $(\lnot, \land, \lor, \to, \leftrightarrow)$ и кванторами $(\forall, \exists)$.
\end{enumerate}

Свободные переменные формулы -- это переменные, которые не находятся под действием кванторов ($\forall$ или $\exists$) внутри этой формулы. То есть, они не ''связаны'' кванторами и могут принимать любые значения из области определения.

Предложения в логике первого порядка -- это формулы, которые не содержат свободных переменных, то есть все переменные в них связаны кванторами. Такие формулы имеют логическое значение (истинность или ложность) в интерпретации.

\subsection{Оценка переменных. Значение терма и формулы в данной структуре при данной оценке. Независимость значения формулы от значений переменных, не являющихся ее параметрами.}

Оценка переменных -- способ присвоения конкретных значений переменным в формуле. По сути это функция $\mu$, которая ставит в соответствие \textit{каждой} переменной какое-то значение.

Значение терма $t$ и формулы $\varphi$ в данной структуре $\MM$ при данной оценке $\mu$: 
\begin{enumerate}
    \item если $t$ -- переменная, то $t$ принимает значение $\mu(t)$
    \item если $t$ -- константный символ $c$, то $t$ принимает значение интерпретации $c$ в $\MM$: $c^\MM$
    \item если $t$ -- функция $f$, применяемая к термам $t_1,\ldots,t_n$, то значение $t$ -- это $f^\MM(v_1,\ldots,v_n)$, где $v_1,\ldots,v_n$ -- это значения термов при данной оценке
    \item если $\varphi$ -- атомарная формула $P(t_1,\ldots,t_n)$, то она истинна, если $(v_1,\ldots,v_n)\in R^\MM$, где $v_1,\ldots,v_n$ -- это значения термов при данной оценке
    \item для сложных формул $\varphi$ используются стандартные логические правила
\end{enumerate}

Независимость значения формулы от значений переменных, не являющихся ее параметрами означает, что если мы изменим значения переменных, которые не являются свободными в данной формуле, то значение формулы останется неизменным. Другими словами, переменные, не являющиеся \textit{свободными} в формуле, не влияют на ее истинностное значение.

\subsection{Значение терма и формулы на наборе элементов структуры. Выразимые в структуре множества (отношения, функции, элементы). Примеры выразимых множеств.}

Значение терма или формулы $\alpha(x_1,\ldots,x_n)$ на наборе элементов $y=(y_1,\ldots,y_n)$ структуры $\MM$ определяется значением функции $\alpha^\MM(y)=[\alpha](\pi + (x_1\to y_1) + \ldots + (x_n\to y_n))$, где $\pi$ -- любая оценка.

\hfill

Выразимые в структуре $\MM$ множества -- это множества $D\subseteq\MM$, которые можно описать с помощью формул логики первого порядка

Примеры:
\begin{enumerate}
    \item пустое множество: $\varphi(x)=(x\not=x)$
    \item носитель структуры $\MM$: $\varphi(y)=(y=y)$
    \item четные числа: $\varphi(z)=\exists a (a\in\N \land a+a=z)$
\end{enumerate}

Выразимые в структуре предикаты -- это предикаты, для которых существуют эквивалентные формулы логики первого порядка

\end{document}
