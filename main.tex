\documentclass[a4paper, 10pt]{article}
\usepackage[unicode=true, colorlinks=true, linkcolor=blue, urlcolor=blue]{hyperref}
\usepackage[T2A]{fontenc}
\usepackage[utf8]{inputenc}
\usepackage[russian]{babel}
%\usepackage{indentfirst}
\usepackage{amssymb}
\usepackage{enumitem}
\usepackage{hyperref}
\usepackage{geometry}
\usepackage{mathtools}
\usepackage{setspace}
\usepackage{tikz}
\usepackage{textcomp}
\usepackage{ stmaryrd }
\usepackage{ulem}
\usepackage{ dsfont }
\usepackage{tikzsymbols}
\usepackage{graphicx}
\usepackage{graphicx}
\graphicspath{{img/}}
\DeclareGraphicsExtensions{.pdf,.png,.jpg}

% Эта штука делает так, что у первого абзаца есть отступ. Почему-то по дефолту в техе все отступы начинаются со второго. Можешь закомментировать и посмотреть, что из этого получится
\usepackage{indentfirst}
\newcommand{\FF}{\mathcal{F}}
\newcommand{\RR}{\mathcal{R}}
\newcommand{\CC}{\mathcal{C}}
\newcommand{\MM}{\mathcal{M}}
\newcommand{\II}{\mathcal{I}}
\newcommand{\NN}{\mathcal{N}}
\newcommand{\UU}{\mathcal{U}}

\newcommand{\dom}{\text{dom}\,}
\newcommand{\rng}{\text{rng}\,}
\newcommand{\pr}{\text{pr}}

\geometry{a4paper,top=2cm,bottom=2cm,left=2cm,right=2cm}
\usepackage{hwspecial}

%$$\includegraphics{}$$

\title{Коллоквиум по дискретной математике 2}
\date{}
\author{Ми (@technothecow)}
\begin{document}

\maketitle

\tableofcontents

\newpage

\section{Логика и машины Тьюринга}

\subsection{Структуры и сигнатуры. Нормальные структуры. Изоморфизм структур.}

Структура -- кортеж множеств $(M, \FF, \RR, \CC)$, где 
\begin{enumerate}
    \item $M$ -- непустое множество, \textit{носитель структуры}
    \item $\FF$ -- множество функций вида $f\colon M^n\to M$
    \item $\RR$ -- множество кортежей из $M$
    \item $\CC$ -- подмножество $M$
\end{enumerate}

Сигнатура -- кортеж попарно непересекающихся множеств $(Fnc,Prd,Cnst)$, где $Fnc$ -- множество функциональных символов с заданной валентностью, $Prd$ -- непустое множество предикатных символов с заданной валентностью и $Cnst$ -- множество константных символов. (просто набор символов)

*$\sigma$-структура (или интерпретация сигнатуры $\sigma$) -- это формально кортеж $\MM = (M,\FF,\RR,\CC,\II)$, где $\II(Fnc)=\FF,~\II(Prd)=\RR$ и $\II(Cnst)=\CC$. Вводим обозначения: $\II(Fnc)=f^\MM$, $\II(Prd)=R^\MM$ и $\II(Cnst)=c^\MM$. Для задания $\sigma$-структуры достаточно только $M$ и $\II$. Фактически, мы придаем значение имеющимся значкам из сигнатуры $\sigma$: берем носитель и говорим, что делают с ним функции и что делают с ним предикаты.

Нормальная структура -- содержащая двувалентный предикатный символ ``=`` $:=~\{(a, a) \in M^2~|~a \in M\}$, где $M$ -- носитель структуры.

% **Инъекция -- отображение такое, что из $f(x_1)=f(x_2)$ следует $x_1=x_2$.

Изоморфизм структур: интепретации $\MM$ и $\NN$ сигнатуры $\sigma$ с носителями $M$ и $N$ соответственно изоморфны если существует биекция $\eta \colon M \to N$ для которой выполняются следующие свойства:
\begin{enumerate}
    \item $\eta(f^\MM(a_1,\ldots,a_n))=f^\NN(\eta(a_1),\ldots,\eta(a_n))$
    \item $(a_1,\ldots,a_n)\in R^\MM \iff (\eta(a_1),\ldots,\eta(a_n))\in R^\NN$
    \item $\eta(c^\MM) = c^\NN$, где $c$ -- один символ
\end{enumerate}

\subsection{Формулы первого порядка данной сигнатуры. Параметры (свободные переменные)
формулы. Предложения.}

Формулы первого порядка -- это выражения в логике первого порядка (предикатной логике), построенные по правилам синтаксиса, установленным для данной сигнатуры.

Формулы первого порядка строятся из термов и предикатов, используя логические связки и кванторы. Основные элементы синтаксиса формул первого порядка:
\begin{enumerate}
    \item Термы: 1) переменные; 2) константы; 3) если $t_1,\ldots,t_n$ - термы, а $f$ - функция с валентностью $n$, то $f(t_1,\ldots,t_n)$ - тоже терм
    \item Атомарные формулы: предикаты, примененные к термам.
    \item Сложные формулы: атомарные формулы, соединенные логическими операциями $(\lnot, \land, \lor, \to, \leftrightarrow)$ и кванторами $(\forall, \exists)$.
\end{enumerate}

Свободные переменные формулы -- это переменные, которые не находятся под действием кванторов ($\forall$ или $\exists$) внутри этой формулы. То есть, они не ''связаны'' кванторами и могут принимать любые значения из области определения. Множество свободных переменных в формуле $\varphi$ обозначается как $FV(\varphi)$. Множество всех переменных в формуле обозначается как $V(\varphi)$.

Предложения в логике первого порядка -- это формулы, которые не содержат свободных переменных, то есть все переменные в них связаны кванторами. Такие формулы имеют логическое значение (истинность или ложность) в интерпретации.

\subsection{Оценка переменных. Значение терма и формулы в данной структуре при данной оценке. Независимость значения формулы от значений переменных, не являющихся ее параметрами.}

Оценка переменных -- способ присвоения конкретных значений переменным в формуле. По сути это функция $\mu$, которая ставит в соответствие \textit{каждой} (в том числе свободной!) переменной какое-то значение.

Значение терма $t$ и формулы $\varphi$ в данной структуре $\MM$ при данной оценке $\mu$: 
\begin{enumerate}
    \item если $t$ -- переменная, то $t$ принимает значение $\mu(t)$
    \item если $t$ -- константный символ $c$, то $t$ принимает значение интерпретации $c$ в $\MM$: $c^\MM$
    \item если $t$ -- функция $f$, применяемая к термам $t_1,\ldots,t_n$, то значение $t$ -- это $f^\MM(v_1,\ldots,v_n)$, где $v_1,\ldots,v_n$ -- это значения термов при данной оценке
    \item если $\varphi$ -- атомарная формула $P(t_1,\ldots,t_n)$, то она истинна, если $(v_1,\ldots,v_n)\in R^\MM$, где $v_1,\ldots,v_n$ -- это значения термов при данной оценке
    \item для сложных формул $\varphi$ используются стандартные логические правила
\end{enumerate}

Независимость значения формулы от значений переменных, не являющихся ее параметрами: для любых оценок $\pi_1,\pi_2$, терма $t$ и формулы $\varphi$ выполняется:
\begin{enumerate}
    \item если для всех $x\in V(t) \colon \pi_1(x)=\pi_2(x)$, тогда $[t](\pi_1)=[t](\pi_2)$
    \item если для всех $x\in FV(\varphi) \colon \pi_1(x)=\pi_2(x)$, тогда $[\varphi](\pi_1)=[\varphi](\pi_2)$
\end{enumerate}

Доказательство:

\begin{enumerate}
    \item индукция по построению терма $t$:
    \begin{enumerate}
        \item если $t=z$, тогда $[t](\pi_1)=\pi_1(z)=\pi_2(z)=[t](\pi_2)$
        \item если $t=f(a_1,\ldots,a_n)$, тогда $[t](\pi_1)=f([a_1](\pi_1),\ldots,[a_n](\pi_1))=f([a_1](\pi_2),\ldots,[a_n](\pi_2))=[t](\pi_2)$ в силу $V(a_i)\subseteq V(t)$ по предположению индукции.
    \end{enumerate}
    \item индукция по построению формулы $\varphi$: 
    \begin{enumerate}
        \item если $\varphi=P(a_1,\ldots,a_n)$, то для каждого терма $a_i$ имеем $V(a_i)\subseteq FV(\varphi)$, поэтому $[\varphi](\pi_1)=P([a_1](\pi_1),\ldots,[a_n](\pi_1))=P([a_1](\pi_2)\ldots,[a_n](\pi_2))=[t](\pi_2)$
        \item если $\varphi=\lnot \psi$, тогда $[\varphi](\pi_1)=1-[\psi](\pi_1)=1-[\psi](\pi_2)=[\varphi](\pi_2)$ по предположению индукции в силу $FV(\varphi)=FV(\psi)$.
        \item если $\varphi=\psi_1\land \psi_2$, тогда по предположению индукции в силу $FV(\psi_i)\subseteq FV(\varphi)$ выполняется $[\varphi](\pi_1)=\min([\psi_1](\pi_1),[\psi_2](\pi_1))=\min([\psi_1](\pi_2),[\psi_2](\pi_2))=[\varphi](\pi_2)$. Случаи других связок аналогичны.
        \item если $\varphi=\forall z\psi$, тогда $[\varphi](\pi_1)=\min\limits_{m\in M}[\psi](\pi_1 + (z\to m))$. Так как $FV(\psi)\subseteq FV(\varphi)\cup\{z\}$, рассмотрим как работает $\pi_1+(z\to m)$ на $FV(\varphi)\cup\{z\}$.
        \begin{enumerate}
            \item если $y\in FV(\varphi)$, то поскольку $z\not\in FV(\varphi)$, $y\not=z$, следовательно $(\pi_1+(z\to m)(y)=\pi_1(y)=\pi_2(y)=(\pi_2+(z\to m))(y)$.
            \item если $y=z$, тогда $(\pi_1+(z\to m)(y)=m=(\pi_2+(z\to m))(y)$.
        \end{enumerate}
        Таким образом, для любого $y\in FV(\psi)$ имеем $(pi_1+(z\to m))(y)=(pi_2+(z\to m))(y)$. По предположению индукции заключаем $[\psi](\pi_1+(z\to m))=[\psi](\pi_2+(z\to m))$, из чего следует $[\varphi](\pi_1)=[\varphi](\pi_2)$. Случай квантора существования аналогичен.
    \end{enumerate}
\end{enumerate}

\subsection{Значение терма и формулы на наборе элементов структуры. Выразимые в структуре множества (отношения, функции, элементы). Примеры выразимых множеств.}

Значение терма или формулы $\alpha(x_1,\ldots,x_n)$ на наборе элементов $y=(y_1,\ldots,y_n)$ структуры $\MM$ определяется значением функции $\alpha^\MM(y)=[\alpha](\pi + (x_1\to y_1) + \ldots + (x_n\to y_n))$, где $\pi$ -- любая оценка.

\hfill

Выразимые в структуре $\MM$ множества -- это множества $D\subseteq\MM$, которые можно описать с помощью формул логики первого порядка

Примеры:
\begin{enumerate}
    \item пустое множество: $\varphi(x)=(x\not=x)$
    \item носитель структуры $\MM$: $\varphi(y)=(y=y)$
    \item четные числа: $\varphi(z)=\exists a (a\in\N \land a+a=z)$
\end{enumerate}

Выразимые в структуре предикаты -- это предикаты, для которых существуют эквивалентные формулы логики первого порядка

\subsection{Значение формулы при изоморфизме структур. Элементарная эквивалентность структур. Изоморфные структуры элементарно эквивалентны.}

*Если $\sigma$-предложение $\varphi$ истинно в $\MM$, то это обозначается так: $\MM\models\varphi$

*Теория в языке сигнатуры $\sigma$ -- это какое-то множество $\sigma$-предложений.

*Модель теории $T$ в языке сигнатуры $\sigma$ -- это такая $\sigma$-структура $\MM$, что все предложения в ней истинны.

*Модель предложения $\varphi$ в языке сигнатуры $\sigma$ -- это модель теории $\{\varphi\}$.

*Теория $\sigma$-структуры $\MM$ -- это все $\sigma$-предложения, истинные в $\MM$. Обозначение: $Th(\MM)$.

\hfill

Элементарная эквивалентность структур: $\sigma$-структуры $\MM$ и $\NN$ эквивалентны если $Th(\MM)=Th(\NN)$. Обозначение: $\MM\equiv\NN$

\hfill

Значение формулы $\varphi$ при изоморфизме $\eta$ структур $\MM$ и $\NN$: для любого $a \in M^n$ и любой формулы $\varphi$ равносильны $\MM \models \varphi(a)$ и $\NN \models \varphi(\eta(a))$.

(?) Доказательство: по определению изоморфизма $\varphi^\NN(\eta(a))=\eta(f^\MM(a))$ и $\eta(True^\MM)=True^\NN$

\hfill

Элементарная эквивалентность изоморфных структур: изоморфные структуры элементарно эквивалентны.

(?) Доказательство: следует из равносильности $\MM \models \varphi(a)$ и $\NN \models \varphi(\eta(a))$.

\subsection{Значение формулы при изоморфизме структур. Сохранение выразимых множеств автоморфизмами структуры. Примеры невыразимых множеств.}

Значение формулы при изоморфизме структур: см. билет 1.5

\hfill

Сохранение выразимых множеств автоморфизмами структуры: семейство выразимых множеств сохраняется между автоморфизмами

(?) Доказательство: пусть $A\subseteq M$ выразимо в $\MM$. Это значит, что $a\in A \iff \MM\models\varphi(a)$. Для автоморфизма $\eta$: $a\in A \iff \MM\models\varphi(a) \iff \NN\models\varphi(\eta(a)) \iff \eta(a) \in \eta(A)$

\hfill

Примеры невыразимых множеств: множество всех простых чисел (для этого необходимо проверять все возможные делители); множество натуральных чисел, являющихся степенью двойки (для этого требуется, например, рекурсия, которой нет).

% Эквивалентность формул первого порядка. [F, с. 4] Лемма о фиктивном кванторе. [F, 10] Общезначимые и выполнимые формулы. Квантор всеобщности и общезначимость. [F, 12]
\subsection{Эквивалентность формул первого порядка. Лемма о фиктивном кванторе. Общезначимые и выполнимые формулы. Квантор всеобщности и общезначимость.}

Эквивалентность формул первого порядка: формулы $\varphi$ и $\psi$ являются эквивалентными, если их значения совпадают в любой интерпретации при любой оценке. Обозначение $\varphi\equiv\psi$.

\hfill

Лемма о фиктивном кванторе: пусть $x$ не лежит в множестве свободных переменных формулы $\varphi$, тогда $\varphi\equiv\forall x\varphi$

Доказательство: $[\forall x\varphi](\pi)=\min\limits_{m \in M}[\varphi](\pi+(x\to m))$. Так как $x\not\in FV(\varphi)$, для всех $y\in FV(\varphi)$ выполнено $(\pi+(x\to m))(y)=\pi(y)$. По лемме о независимости значения формулы от значений переменных, не являющихся ее параметрами (см. билет 1.3), заключаем $[\varphi](\pi+(x\to m))=[\varphi](\pi)$ для всех $m \in M$. Отсюда $[\forall x\varphi](\pi)=\min\limits_{m\in M} [\varphi](\pi + (x\to m)) = \min\limits_{m\in M} [\varphi](\pi) = [\varphi](\pi)$

\hfill

Общезначимая формула -- формула, истинная при любой интерпретации и оценке.

Выполнимая формула -- формула, для которой существует интерпретация и оценка, в которой она истинна.

\hfill

Квантор всеобщности и общезначимость: формула $\varphi$ общезначима $\iff$ формула $\forall y\varphi$ общезначима

Доказательство:
\begin{enumerate}
    \item слева направо: формула общезначима, значит для любых оценок равна единице, в частности для оценок вида $(\pi + (y\to m))$ для всех $m \in M$, поэтому $[\forall y\varphi](\pi)=1$
    \item справа налево: $\forall y\varphi$ общезначима, значит для любых оценок $[\varphi](\pi + (y\to m))=1$ для всех $m\in M$. Однако для любой оценки $\pi$ имеем $\pi=(\pi+(y\to\pi(y)))$, поэтому $[\varphi](\pi)=1$ для всех оценок $\pi$.
\end{enumerate}

% Основные эквивалентности логики первого порядка [F, 24]. Замена подформулы на эквивалентную. [F, 26, 30]
\subsection{Основные эквивалентности логики первого порядка. Замена подформулы на эквивалентную.}

Основные эквивалентности логики первого порядка для произвольных $\varphi$ и $\psi$:

\begin{enumerate}
    \item Пусть $x$ не является параметром $\psi$, тогда $\forall\{\exists\} x (\varphi \land\{\lor\} \psi) \equiv \forall\{\exists\}x\varphi\land\{\lor\}\psi$ (итого 4 равенства)
    \item $\forall x (\varphi\land\psi) = \forall x\varphi \land \forall x\psi$
    \item $\forall x (\varphi\lor\psi) = \forall x\varphi \lor \forall x\psi$
    \item $\lnot\forall x\varphi \equiv \exists x \lnot \varphi$
    \item $\lnot\exists x\varphi \equiv \forall x \lnot \varphi$
\end{enumerate}

Доказательство: см. \href{https://drive.google.com/file/d/1V3QNOfANqZbiacI0d_2aLy8l1gZP0rTc/view}{first-order} стр. 7-8

\hfill

Пусть $\varphi$ -- какая-то формула, $\varphi\equiv\varphi'$, тогда замена $\varphi$ на $\varphi'$ эквивалентна в случаях использования логического и, или, не, импликации, ''тогда и только тогда'', квантора всеобщности и существования.

Доказательство: 1-6) тривиально; 7) для любой оценки $\pi$ и $m\in M$ имеем $[\varphi](\pi+(x\to m))=[\varphi'](\pi+(x\to m))$, отсюда $[\forall x\varphi](\pi)=\min\limits_{m \in M}[\varphi](\pi+(x\to m))=\min\limits_{m \in M}[\varphi'](\pi+(x\to m))=[\forall x\varphi'](\pi)$; 8) аналогично 7

\hfill

Замена подформулы на эквивалентную: пусть $\varphi\equiv\varphi'$ и $\psi'$ была получена путем замены вхождений $\varphi$ в $\psi$ на $\varphi'$, тогда $\psi\equiv\psi'$.

Доказательство: достаточно рассмотреть случай одного вхождения. Индукция по построению $\psi$. Рассмотрим один из случаев: пусть $\psi = \theta_1\to\theta_2$, тогда подформула либо совпадает с формулой, либо вхождение будет в $\theta_1$ или $\theta_2$. Применим к $\theta_i$ предположение индукции и получим, например, $\psi'=\theta_1'\to\theta_2$ и $\theta_1\equiv\theta_1'$, далее используем подходящее утверждение из предыдущей леммы (та, что над этой) и заключаем $\psi=\psi'$.

\subsection{Пропозциональные формулы и задаваемые ими булевы функции. Тавтологии первого порядка.}

Пропозициональная формула -- формула, построенная из пропозициональных переменных (простых букв) с помощью булевых связок.

Каждая пропозициональная формула задаёт булеву функцию, так как для каждого набора значений переменных (истина или ложь) формула принимает одно определённое значение (истина или ложь). То есть, если у вас есть пропозициональная формула $A$ с переменными $p$ и $q$, можно построить таблицу истинности, которая покажет значение формулы для всех возможных значений $p$ и $q$.

Тавтология -- это формула, истинная при любых значениях ее переменных. Любая тавтология общезначима.

% Лемма о корректной подстановке. [F, 73]
\subsection{Лемма о корректной подстановке.}

*Терм $t$ свободен для переменной $x$ в формуле $\varphi$, если при подстановке терма $t$ вместо переменной $x$ в формуле $\varphi$ не происходит никаких изменений значений других свободных переменных. Иными словами, терм $t$ можно подставить на место $x$ в $\varphi$ без появления новой привязки переменных, которая может изменить интерпретацию формулы. Обозначение: $t-x-\varphi$.

*Замена $y$ на $x$ в формуле $\varphi$ обозначается как $\varphi(y/x)$

Лемма о корректной подстановке: в любой интерпретации при любой оценке $\pi$ для всех $\varphi$ - формул, $t,s$ - термов, и $x$ - переменной, если $t-x-\varphi$, то выполняется:

$$ [s(t/x)](\pi) = [s](\pi + (x \to [t](\pi))) \text{ и } [\varphi(t/x)](\pi)=[\varphi](\pi + (x\to [t](\pi))) $$

TODO: доказательство

% Понятие корректной подстановки («терм свободен для переменной в формуле»). Пример некорректной подстановки. Лемма о корректной подстановке (без доказательства). [F, 73] Переименование связанной переменной. [F, 16, 18]. Общезначимость формул вида ∀xϕ → ϕ(t/x) и ϕ(t/x) → ∃xϕ в случае корректной подстановки. [F, 74]
\subsection{Понятие корректной подстановки («терм свободен для переменной в формуле»). Пример некорректной подстановки. Лемма о корректной подстановке (без доказательства). Переименование связанной переменной. Общезначимость формул вида $\forall x\varphi \to \varphi(t/x)$ и $\varphi(t/x) \to \exists x \varphi$ в случае корректной подстановки.}

см. билет 1.10

Пример некорректной подстановки: возьмем формулу $\varphi(x,y)=\forall y (P(x,y))$ и терм $t=y$. Подставляем: $\varphi(x/t,y)=\forall y (P(y,y))$. Смысл формулы изменен т.к. терм не свободен для переменной в формуле.

\hfill

Переименование связанной переменной:

Лемма 1. Пусть $y\not\in V(\varphi)$ (т.е. $y$ нет в $\varphi$), тогда $\forall x\varphi \equiv \forall y\varphi(y/x)$.

Лемма 2. Для любого терма $t$ и любой формулы $\varphi$, если $y \not\in V(\varphi)$, то для любой оценки $\pi$ верно: $[t(y/x)](\pi)=[t](\pi + (x \to \pi(y)))$ и $[\varphi(y/x)](\pi)=[\varphi](\pi + (x \to \pi(y)))$

\hfill

\begin{enumerate}
    \item $\forall x\varphi \to \varphi(t/x)$, если $t$ свободен для $x$ в $\varphi$
    \item $\varphi(t/x) \to \exists x \varphi(x)$, если $t$ свободен для $x$ в $\varphi$
\end{enumerate}

\hfill

TODO: дописать доказательства

% Переименование связанной переменной (без доказательства). [F, 16] Теорема о предваренной нормальной форме. [F, 36]
\subsection{Переименование связанной переменной (без доказательства). Теорема о предваренной нормальной форме.]}

Переименование связанной переменной:

Лемма 1. Пусть $y\not\in V(\varphi)$ (т.е. $y$ нет в $\varphi$), тогда $\forall x\varphi \equiv \forall y\varphi(y/x)$.

Лемма 2. Для любого терма $t$ и любой формулы $\varphi$, если $y \not\in V(\varphi)$, то для любой оценки $\pi$ верно: $[t(y/x)](\pi)=[t](\pi + (x \to \pi(y)))$ и $[\varphi(y/x)](\pi)=[\varphi](\pi + (x \to \pi(y)))$

\hfill

*Предваренная формула -- такая, что имеет кванторы только в кванторном префиксе в начале формулы.

Теорема о предваренной нормальной форме: для любой формулы найдется эквивалентная ей предваренная.

Доказательство: индукция по построению. Разберем все случаи:

\begin{enumerate}
    \item Если формула атомарная, то она уже предваренная.
    \item Если формула начинается с квантора, то по предположению индукции заменяем формулу под этим квантором на эквивалентную предваренную.
    \item Если формула начинается с отрицания, то по предположению индукции заменяем формулу под отрицанием на эквивалентную предваренную и проносим отрицание вовнутрь, переменяя кванторы.
    \item Если в формуле главная связка бинарная, то по предположению индукции заменяем формулы под связкой на эквивалентные предваренные и переименовываем связанные переменные так, чтобы все кванторы можно было вынести наружу и выносим их.
\end{enumerate}

% Понятие теории первого порядка. Примеры содержательных теорий. Модель теории. Логическое (семантическое) следование (для теорий и предложений). [ВШ-2, с. 187].
\subsection{Понятие теории первого порядка. Примеры содержательных теорий. Модель теории. Логическое (семантическое) следование (для теорий и предложений).}

Теория первого порядка -- логическая система, включающая в себя сигнатуру (набор символов, включающий константы, функции и предикаты), аксиомы (набор утверждений или формул, принимаемых без доказательств) и правила вывода (правила, по которым из аксиом и других утверждений можно выводить новые утверждения)

\hfill

Примеры содержательных теорий:

\begin{enumerate}
    \item Теория групп:
    \begin{enumerate}
        \item Сигнатура: бинарная операция $*$ и константа $e$
        \item Аксиомы: ассоциативность, существование нейтрального элемента, существование обратного элемента
    \end{enumerate}
    \item Теория колец:
    \begin{enumerate}
        \item Сигнатура: две бинарные операции: $+$ и $*$ и константы $0$ и $1$.
        \item Аксиомы: дистрибутивность, ассоциативность, коммутативность, существование обратного элемента по сложению
    \end{enumerate}
\end{enumerate}

Модель теории -- это интерпретация сигнатуры, в которой все аксиомы теории истинны. Например, для теории групп это множество целых чисел с операцией сложения и нулем.

\hfill

Логическое следование -- отношение между формулами и теориями, которое говорит, что если истинны определенные формулы, то и другие формулы истинны.

Для теорий: Теория $T$ логически следует из множества аксиом $A$, если любая модель $A$ также является моделью $T$.

Для предложений: Предложение $\varphi$ логически следует из теории $T$ ($T\models\varphi$), если $\varphi$ истинно в каждой модели $T$.

\subsection{Исчисление предикатов с равенством (в гильбертовской форме). Теорема о полноте и корректности исчисления предикатов (без доказательства). Теорема о компактности в двух формах: про выполнимость теории и про логическое следование из теории.}

Исчисление предикатов с равенством -- это система логики первого порядка, включающая равенство как основной предикат. В гильбертовской форме исчисления предикатов используются аксиомы и правила вывода.

Аксиомы для равенства:

\begin{enumerate}
    \item Рефлексивность: $\forall x (x=x)$
    \item Симметричность: $\forall x \forall y (x=y \to y=x)$
    \item Транзитивность: $\forall x\forall y\forall z (x=y\land y=z \to x=z)$
    \item Замена в формулах: если $t$ -- терм, а $P$ -- предикат, то $\forall x\forall y (x=y \to (P(x)\leftrightarrow P(y)))$
\end{enumerate}

Общие аксиомы и правила вывода:

\begin{enumerate}
    \item Аксиомы логики первого порядка
    \item Правило Modus Ponens: из $\varphi$ и $\varphi\to\psi$ следует $\psi$
    \item Правило обобщения: из $\varphi$ следует $\forall x\varphi$, если $x$ не свободная в $\varphi$
\end{enumerate}

Теорема о полноте и корректности исчисления предикатов: если $\varphi$ логически следует из $A$, тогда и только тогда $\varphi$ выводима из $A$ в исчислении предикатов.

\hfill

Теорема о компактности: если любая конечная подсистема множества предложений имеет модель, то и все множество имеет модель.

Теорема о компактности в форме про выполнимость теории: если каждое конечное подмножество множества формул $T$ выполнимо, то и все множество $T$ выполнимо.

Теорема о компактности в форме про логическое следование из теории: формула $\varphi$ логически следует из теории $T$ тогда и только тогда, когда $\varphi$ логически следует из некоторого конечного подмножества теории $T$.

TODO: дополнить доказательствами

\subsection{Теорема компактности (без доказательства). Любой пример применения.}

см. билет 1.14

Пример: хотим показать, что существует бесконечное множество.

Пусть $T$ -- это теория, содержащая набор формул $F=\{\varphi_n \colon n \in \N\}$, где $\varphi_n$ утверждает, что в нашем множестве существует как минимум $n$ различных элементов. Любое конечное подмножество $F$ выполнимо в модели потому что можно найти конечное число элементов, принадлежащих множеству. Применяем теорему компактности: раз каждое подмножество $F$ имеет модель, то и все множество $F$ имеет модель, значит существует модель, содержащая бесконечно много элементов.

% Одноленточная машина Тьюринга (допустимо неформальное определение с лентой и головкой). Сложение натуральных чисел (при унарном и бинарном кодировании). [Крп]
\subsection{Одноленточная машина Тьюринга (допустимо неформальное определение с лентой и головкой). Сложение натуральных чисел (при унарном и бинарном кодировании).}

Одноленточная машина Тьюринга — это теоретическая модель вычислений, состоящая из следующих частей: лента (бесконечная в обе стороны, разделенная на ячейки, каждая из которых может хранить один символ из конечного алфавита, который обычно содержит спец.символ "пусто": $\#$), головка для чтения/записи (устройство, которое может перемещаться влево или вправо по ленте, считывать символы с ленты и записывать символы на ленту), множество состояний (конечное множество состояний, одно из которых является начальным, а одно или несколько могут быть конечными) и таблица переходов (определяет правила, по которым машина переходит из одного состояния в другое, в зависимости от символа под головкой)

Сложение натуральных чисел в унарном виде: очевидно

Сложение натуральных чисел в бинарном виде: вводим понятие дополнительных переменных в состоянии, типа чтобы хранить $n$ бит, нам понадобится в $n$ раз больше состояний. Тогда просто складываем в столбик, поддерживая в данный момент "в уме" (а точнее в дополнении к состоянию) переполнения

\subsection{Многоленточная машина Тьюринга (допустимо неформальное определение с лентами и головками). Удвоение входного слова за линейное время.}

Многоленточная машина Тьюринга — это расширение классической машины Тьюринга, у которой есть несколько лент и несколько головок для чтения/записи. Каждая лента бесконечна в обе стороны и содержит свой собственный алфавит символов.

Удвоение входного слова за линейное время: копируем символы пока не дойдем до решетки. Как дошли до решетки, идем на верхней ленте влево в начало слова и повторяем процедуру.

\subsection{Конфигурации одноленточной и многоленточной машин Тьюринга. Меры сложности «время» и «зона» и их соотношение в обоих случаях.}

Конфигурация машины Тьюринга -- это описание текущего состояния машины, которое включает состояние машины, содержимое ленты (лент), позиция головки (головок).

Время выполнения (или временная сложность) алгоритма на машине Тьюринга -- это количество шагов, которые машина делает для выполнения задачи. Временная сложность оценивается в зависимости от размера входных данных $n$.

Зона выполнения (или пространственная сложность) алгоритма на машине Тьюринга -- это количество ячеек ленты, которые машина использует для выполнения задачи.

\hfill

Существуют \href{https://www.cs.bu.edu/faculty/gacs/courses/cs535/papers/HennieStearns66.pdf}{работы}, которые показывают, что алгоритм, выполненный на МТ из $k$ лент эмулируется за $T\log T$ на двуленточной МТ.

Многоленточные машины Тьюринга более эффективны по времени (например, задача удвоения входного слова) по сравнению с одноленточными машинами, так как позволяют параллельно обрабатывать несколько лент и перемещаться быстрее по необходимым данным. Однако, пространственная сложность остаётся асимптотически такой же, как и для одноленточных машин.

\subsection{Сокращение ленточного алфавита и его цена.}

См. страницы 21-24 в \href{https://static42.fileskachat.com/download/6/c/33377_672968aac1311878215538743df4fbd9.pdf}{''Введении в сложность вычислений'' Крупского}

\subsection{Сокращение числа лент и его цена.}

См. страницы 24-27 в \href{https://static42.fileskachat.com/download/6/c/33377_672968aac1311878215538743df4fbd9.pdf}{''Введении в сложность вычислений'' Крупского}

\section{Вычислимость}

% Вычислимые функции (при интуитивном понимании алгоритма). Разрешимые и перечислимые множества. Связь конечности, разрешимости и перечислимости. Разрешимые множества под действием операций алгебры множеств и декартова произведения. [R, 7, 9, с. 2–4]
\subsection{Вычислимые функции (при интуитивном понимании алгоритма). Разрешимые и перечислимые множества. Связь конечности, разрешимости и перечислимости. Разрешимые множества под действием операций алгебры множеств и декартова произведения.}

Вычислимая функция -- это такая частичная функция $f\colon\N\to\N$, что для нее существует программа (алгоритм), которая на любом входе $x\in\dom f$ выписывает $f(x)$, а иначе зацикливается.

Разрешимое множество -- такое множество, чья характеристическая функция (функция, которая ест элемент и выплевывает единицу если элемент в множестве и ноль иначе) вычислима.

Перечислимое множество -- такое множество, для которого есть программа, которая последовательно выписывает все элементы множества и только их. Для каждого элемента множества должно существовать $k\in\N$, что после $k$-ого шага элемент будет выписан.

\hfill

Связь конечности, разрешимости и перечислимости: 1) конечно, значит разрешимо; 2) разрешимо, значит перечислимо.

Доказательство: 1) конечно, значит можно пронумеровать элементы $\{a_1,...,a_n\}$. Искомая характеристическая функция равна дизъюнкции (логическому или) булевских значений $x=a_1\lor x=a_2 \lor \ldots \lor x=a_n$. Для пустой функции всегда возвращаем ноль, что также вычислимо.

2) перебираем все натуральные числа и выводим текущее если характеристическая функция вернула единицу

\hfill

Разрешимые множества под действием операций алгебры множеств и декартова произведения: $A,B$ -- разрешимы $\implies$ разрешимы: $A\cup B, A\cap B, A\times B, \overline{A}, \overline{B}$

Доказательство: выразим характеристические функции: $\chi_{A\cup B}(x)=\max(\chi_A(x),\chi_B(x))$, и т.д.

\subsection{Перечислимые множества под действием операций алгебры множеств, декартова произведения и проекции. Теорема Поста.}

Перечислимые множества под действием операций алгебры множеств, декартова произведения и проекции: $A,B$ -- перечислимы $\implies$ перечислимы: $A\cup B, A\cap B, A\times B, \text{pr}^i A, \text{pr}^i B$.

Доказательство: перечислимость $A\cup B$: просто выводим числа по очереди; перечислимость $A\cap B$: по очереди выполняем по шагу алгоритмов $A$ и $B$ и когда получаем очередной элемент $a_i$ выводим его только если нам уже попадался равный ему $b_j$. Аналогично поступаем с новыми элементами из $B$; перечислимость $A\times B$: по очереди выполняем по шагу алгоритмов для $A$ и $B$ и когда получаем очередной элемент $a_i$ выписываем пары со всеми до этого полученными $b_1,\ldots,b_k$. Аналогично поступаем и для $B$; перечислимость проекции: просто для каждого нового $a=(a_1,\ldots,a_n)$ выводим $a_i$.

\hfill

Теорема Поста: множество разрешимо $\iff$ его дополнение и оно само перечислимо.

Доказательство: 1) слева направо следует из леммы о связи конечности, разрешимости и перечислимости (билет 2.1)

2) справа налево доказывается с помощью следующего вычислимого алгоритма: будем выполнять по очереди по одному шагу алгоритма для множества и его дополнения. Рано или поздно в первом или втором появится наш проверяемый элемент

% Теорема о графике вычислимой функции. Перечислимость образа и прообраза множества под действием вычислимой функции. [R, 13, 14] 
\subsection{Теорема о графике вычислимой функции. Перечислимость образа и прообраза множества под действием вычислимой функции.}

Теорема о графике вычислимой функции: функция вычислима $\iff$ ее график перечислим (то есть множество пар $(x,f(x))$)

Доказательство: 1) справа налево: просто ждем пока выдаст нужную пару
2) слева направо: переберем все пары $(x,k)\in\N$. $x$ -- значение, $k$ -- количество шагов, которые проделываются для вычисления $x$. Таким образом, если за конечное число шагов значение вычисляется, мы выведем пару.

\hfill

Перечислимость образа и прообраза множества под действием вычислимой функции: пусть множество $A$ -- перечислимо и $f$ -- вычислимая функция. Тогда $f(A)$ и $f^{-1}(A)$ перечислимы.

Доказательство: пусть $G\subseteq\N\times\N$ -- график $f$, тогда множества $M_1=G\cap(A\times\N)$ и $M_2=G\cap(\N\times A)$ перечислимы так как являются пересечением двух перечислимых множеств. Заметим, что $f(A)=\text{pr}^2 M_1$ и $f^{-1}(A)=\text{pr}^1 M_2$.

% Непустые перечислимые множества суть, в точности, области значений вычислимых тотальных функций. [R, 15, 8]
\subsection{Непустые перечислимые множества суть, в точности, области значений вычислимых тотальных функций.}

Лемма: множество $A$ перечислимо $\iff$ $A=\varnothing$ или $\exists f\colon \N\to A$, что $f$ -- тотальная и $\text{rng}\, f=A$.

Доказательство: 1) справа налево: все элементы $A$ выпишет программа, последовательно вычисляющая $f(0),f(1),\ldots$ (вычисление $f(n)$ всегда заканчивается за конечное количество шагов ибо $f$ тотальная и вычислимая).

2) Пусть элементы $A$ выписывает программа $p$. Тогда пусть $m$ -- число шагов в программе $p$ до вывода первого числа. Определим $f$ следующим образом: $f(x)=$последнему числу после $m+x$ шагов. Докажем, что любое $x\in A$ лежит в образе $f$. Для $x$ должно существовать такое $k\in\N$, что после $k$ шагов $x$ выводится программой $p$. Тогда $f(k-m)=x$.

\hfill

Следствие: если $f$ вычислима, тогда $\dom f$ и $\rng f$ перечислимы.

Доказательство: следует из перечислимости образа и прообраза множества под действием вычислимой функции (см. билет 2.3): $\dom f=f^{-1}(\N),\,\rng f=f(\N)$.

% Полуразрешимость. Перечислимые множества суть, в точности, области определения вычислимых функций. [R, 15, 17]
\subsection{Полуразрешимость. Перечислимые множества суть, в точности, области определения вычислимых функций.}

*Полухарактеристическая функция $\varphi$ множества $A$ задается $\varphi = \begin{cases}
    1,&\text{если } x\in A \\ \text{неопр.,}&\text{иначе}
\end{cases}$ 

Полуразрешимое множество -- такое, что его полухарактеристическая функция вычислима.

Лемма: множество перечислимо $\iff$ множество полуразрешимо

Доказательство: 1) слева направо: если перечислимо $A$, то перечислимо и $A\times\{1\}=\Gamma(\varphi)$. По теореме о графике вычислимой функции (см. билет 2.3), $\varphi$ вычислима.

2) справа налево: если $\varphi$ вычислима, то $A=\dom\varphi$ перечислима по следствию (см. билет 2.4)

% Перечислимые множества суть, в точности, проекции разрешимых. Теорема о свойствах, равносильных перечислимости (доказательство на основе утверждений предшествующих вопросов). [R, 18, 19]
\subsection{Перечислимые множества суть, в точности, проекции разрешимых. Теорема о свойствах, равносильных перечислимости (доказательство на основе утверждений предшествующих вопросов).}

Перечислимые множества в точности проекции разрешимых: множество $A\subseteq\N^n$ перечислимо $\iff$ $\exists B\subseteq\N^{n+1}$ разрешимое, что $A=\pr^1(B)$.

Доказательство: 1) справа налево: $B$ разрешимо $\implies$ $B$ перечислимо $\implies$ $\pr^1(B)=A$ перечислимо

2) слева направо: возьмем перечисляющую элементы $A$ программу $p$. Пусть $B = \{(x,k)\in\N^{n+1} \mid \text{программа $p$ выписывает $x$ на шаге $k$}\}$. Заметим, что построенное множество отвечает требованиям: $B$ действительно разрешимо (на входе $(x,k)$ запустим $k$ шагов $p$ и если вывелось $x$, то элемент лежит, иначе нет) и $A=\pr^1(B)$ (т.к. для каждого $x\in A \exists k\in\N$ -- такое, что за $k$ шагов программы $p$ выведется $x$).

\hfill

Пусть $A\subseteq\N$, тогда следующее равносильно:
\begin{enumerate}
    \item $A$ перечислимо
    \item $\exists f\colon\N\to\N$ - вычислимая частичная, что $A=\dom f$
    \item $\exists f\colon\N\to\N$ - вычислимая частичная, что $A=\rng f$
    \item $A=\varnothing$ или $\exists f\colon\N\to\N$ - вычислимая тотальная, что $A=\rng f$
    \item $\exists B\subseteq\N^2$ - разрешимое, что $A=\pr^1(B)$
\end{enumerate}

Доказательство: 1<->5) см. лемму выше; 1<->4) см. билет 2.4 (лемма); 1->2) см. билет 2.5 (берем полухарактеристическую функцию); 2->1) см. билет 2.4 (следствие); 4->3) очев.; 3->1) см. билет 2.4 (следствие);

% Универсальная вычислимая функция (в классе вычислимых функций N p→ N). T-Предикаты. Неразрешимость проблем самоприменимости и остановки. [R, 7–8, 26]
\subsection{Универсальная вычислимая функция (в классе вычислимых функций $\N\stackrel{p}{\to}\N$). T-Предикаты. Неразрешимость проблем самоприменимости и остановки.}

Универсальная вычислимая функция -- такая $U\colon\N^2\to\N$, если она вычислима и для любой вычислимой функции $f$ существует индекс $i$ такой, что $U_i=f$.

\hfill

T-Предикат: пусть $U$ - у.в.ф. и $\UU$ - программа, вычисляющая $U$, тогда определим множество $T=\{(n,x,k) \mid \text{алгоритм $\UU$ останавливается на входе $(n,x)$ за $k$ шагов}\}$. Т-Предикатом называется функция $T(n,x,k):=(n,x,k)\in T$.

\hfill

Неразрешимость проблемы самоприменимости: невозможно создать алгоритм, определяющий, завершится ли программа на собственном коде.

Доказательство: если существует такой алгоритм $p(x)$, возвращающий ноль если программа $x$ зацикливается на вводе $x$ и единицу иначе, то существует программа $f(x)=\begin{cases}
    \text{зацикливается},&\text{если $p(x)=1$} \\ \text{завершается},&\text{если $p(x)=0$}
\end{cases}$. Рассмотрим случаи: если $p(x)=0$, то по определению $f$ зацикливается, но $f(f)$ завершается; если $p(x)=1$, то по определению $f$ завершается, но $f(f)$ зацикливается. Противоречие.

\hfill

Неразрешимость проблемы остановки: нет алгоритма $g$, который бы определял, завершится ли программа на данном входе.

Доказательство: если бы такой алгоритм существовал, то существовал бы и алгоритм $p(x)=g(x,x)$, проверяющий самоприменимость, но такого алгоритма нет.

% Неразрешимость проблем самоприменимости и остановки. Примеры перечислимого неразрешимого и неперечислимого множеств. [R, 26, 27]
\subsection{Неразрешимость проблем самоприменимости и остановки. Примеры перечислимого неразрешимого и неперечислимого множеств.}

Неразрешимость проблем самоприменимости и остановки: см. билет 2.7

\hfill

Пример перечислимого неразрешимого множества: пусть $U$ - у.в.ф., $d(x)=U(x,x)$ тогда $K=\{x\in\N \mid d(x)\text{ - определено}\}$

Доказательство: 1) перечислимость следует из того, что $K=\dom d$ -- вычислимой функции
2) предположим, что $K$ -- разрешимо, тогда определим вычислимую функцию $f(x)=\begin{cases}
    0,&x\not\in K \\ \text{неопр.,}&x\in K
\end{cases}$. Существует $n$, что $U_n = f$. Тогда рассмотрим, лежит ли $n$ в $K$: если да, то $d(n)$ не определено, значит $n\not\in K$; если нет, то $d(n)=0$ - определено, значит $n\in K$. В обоих случаях противоречия, значит предположение ложно.

\hfill

Пример неперечислимого множества: множество $\overline{K}$ -- если бы оно было перечислимо, то по теореме Поста (см. билет 2.2) $K$ было бы разрешимо, что неправда.

% Пример вычислимой функции, не имеющей вычислимого тотального продолжения. Область определения вычислимой функции, не имеющей вычислимого тотального продолжения, перечислима, но не разрешима. [R, 33, 36]
\subsection{Пример вычислимой функции, не имеющей вычислимого тотального продолжения. Область определения вычислимой функции, не имеющей вычислимого тотального продолжения, перечислима, но не разрешима.}

Пример вычислимой функции, не имеющей вычислимого тотального продолжения: пусть $U$ - у.в.ф., тогда $d(x)=U(x,x)$ - искомый пример.

Доказательство: 1) $d$ - вычислима

2) Пусть $g$ продолжает $d$, тогда существует вычислимая тотальная $h(x)=g(x)+1$. Для $h$ существует $n$, что $U_n=h$. Разберем случаи: если $n\not\in\dom d$, тогда не определено $U(n,n)$, но $U(n,n)=U_n(n)=h(n)$ определено, значит $n\in\dom d$, тогда $d(n)=U(n,n)=U_n(n)=h(n)=g(n)+1=d(n)+1$ - противоречие.

\hfill

Область определения вычислимой функции, не имеющей вычислимого тотального продолжения, перечислима, но не разрешима: Пусть вычислимая функция $f$ не имеет вычислимого тотального продолжения, тогда $\dom f$ перечислимо, но не разрешимо.

Доказательство:

1) перечислимость из следствия (см. билет 2.4)

2) от противного: пусть $\dom f$ разрешимо, тогда существует характеристическая функция $g$. Определим $h(x)=\begin{cases}
    f(x),&\text{если }g(x)=1 \\ 0,&\text{если }g(x)=0
\end{cases}$. Таким образом мы получили вычислимое тотальное продолжение, противоречие.

% Невозможность универсальной вычислимой тотальной функции. [ВШ-3, 2.2, т. 8]
\subsection{Невозможность универсальной вычислимой тотальной функции.}

Невозможность универсальной вычислимой тотальной функции: тотальной у.в.ф. не может быть.

Доказательство: от противного: пусть $U$ - тотальная у.в.ф., тогда возьмем диагональ $d(x)=U(x,x)$ и построим $f(x)=d(x)+1$ - тотальная вычислимая функция. Значит существует $n$, что $U_n=f$. Рассмотрим значение $f(n)$: $f(n)=U_n(n)=U(n,n)=d(n)$, но $f(n)=d(n)+1$ по определению, противоречие.

% Пример непересекающихся перечислимых множеств, не отделимых никаким разрешимым множеством. [R, 37, 38]
\subsection{Пример непересекающихся перечислимых множеств, не отделимых никаким разрешимым множеством.}

*Сначала нужно решить упражнение: существует вычислимая функция $f$, не имеющая вычислимого тотального продолжения, т. ч. $\rng f = \{0, 1\}$.

Доказательство: пусть $U$ - у.в.ф. и $d(x)=U(x,x)$. Определим $f(x)=\begin{cases}
    0,&d(x)=0 \\
    1,&d(x)\not=0 \\
    \text{неопр.,}&d(x)\text{ неопр.}
\end{cases}$. Если бы было вычислимое тотальное продолжение $f$, тогда существовало бы и тотальное продолжение $d(x)$.

\hfill

*Отделимость: множество $C$ отделяет $A$ от $B$, если $A\subseteq C$ и $B\subseteq \overline{C}$

Пример непересекающихся перечислимых множеств, не отделимых никаким разрешимым множеством: рассмотрим $f$ из упражнения выше и положим $A=f^{-1}(1)$ и $B=f^{-1}(0)$.

Доказательство: 1) непересекаемость очев.

2) перечислимость из теоремы о графике вычислимой функции (см. билет 2.3)

3) неотделимость разрешимой функцией: если разрешимое $C$ отделяет $A$ и $B$, тогда вычислимая тотальная характеристическая функция $g$ множества $C$ продолжает $f$, чего не может быть, противоречие.

% Главная универсальная вычислимая функция. Вычислимое биективное кодирование пар натуральных чисел. Построение главной у. в. ф. с помощью произвольной у. в. ф. [R, с. 6–7; 23, 41, 42]
\subsection{Главная универсальная вычислимая функция. Вычислимое биективное кодирование пар натуральных чисел. Построение главной у.в.ф. с помощью произвольной у.в.ф.}

Главная универсальная вычислимая функция -- такая частичная вычислимая $U\colon\N^2 \to\N$, что для любой частичной вычислимой функции $F\colon\N^2\to\N$ существует вычислимая тотальная функция $s\colon\N\to\N$, что $F_i=U_{s(i)}$

\hfill

Вычислимое биективное кодирование пар натуральных чисел: пусть $\langle \cdot,\cdot \rangle \colon \N^2 \to \N$ -- произвольная тотальная биекция. Определим тотальные функции $\pi_1$ и $\pi_2$, что $\pi_1(\langle n_1,n_2 \rangle)=n_1$ и $\pi_2(\langle n_1,n_2 \rangle)=n_2$. Функции $\pi_1$ и $\pi_2$ вычислимы.

Доказательство: опишем алгоритм вычисления $\pi_1$ (для $\pi_2$ аналогично). Для заданного $n\in\N$ перебираем все пары $(a,b)\in\N^2$ пока не найдем такого, что $\langle a,b \rangle = n$ и вернем $a$. Мы найдем такую пару так как функция - тотальная биекция.

\hfill

Построение главной у.в.ф. с помощью произвольной у.в.ф.: \textit{редакторское примечание}: построение совсем нетривиальное, просто внимательно следим за руками

Построение: пусть $U$ - у.в.ф.. Определим нашу г.у.в.ф. так: $W(n,x)=U(\pi^1(n),\langle \pi^2(n),x \rangle)$. 

Проверим, что она г.у.в.ф.: 1) вычислимость: мы берем вычислимую функцию и подставляем вычислимые аргументы, все ок.

2) Пусть $V\colon\N^2\to\N$ - какая-то вычислимая функция. Зададим еще одну функцию на основе $V$: $V'(x)=V(\pi^1(x),\pi^2(x))$, она тоже вычислимая, тогда для нее существует какое-то $l$, что $U_l=V'$. И последнее: для $V$ искомая $s(n)=\langle l,n \rangle$, она вычислимая тотальная.

Теперь магия: $W(s(n),x)=W(\langle l,n\rangle,x)=U(\pi^1(\langle l,n \rangle),\langle \pi^2(\langle l,n \rangle),x \rangle = U(l,\langle n,x \rangle) = U_l(\langle n,x \rangle) = V'(\langle n,x \rangle) = V(\pi^1(\langle n,x \rangle),\pi^2(\langle n,x \rangle)) = V(n,x)$

% Теорема Клини о неподвижной точке. [R, 44]
\subsection{Теорема Клини о неподвижной точке}

Теорема: пусть $U$ - г.у.в.ф., $f\colon\N\to\N$ - тотальная вычислимая функция, тогда существует такое $n$, что $U_n=U_{f(n)}$

Доказательство: пусть $V(k,x)=U(U(k,k),x)$ - вычислимая тотальная. Из главности у.в.ф. найдется тотальная $s\colon\N\to\N$, что $U(s(k),x)=V(k,x)=U(U(k,k),x)$. Композиция $f$ и $s$ тоже вычислима, поэтому существует $t\in\N$, что $U_t=f\circ s$. Имеем $U(s(t),x)=V(t,x)=U(U(t,t),x)=U(U_t(t),x)=U((f\circ s)(t),x)=U(f(s(t)),x)$. Искомое $n=s(t)$.

% Бесконечность множества неподвижных точек в смысле теоремы Клини. Теорема о рекурсии как следствие теоремы Клини. Пример применения теоремы о рекурсии. [R, 44, 46, 47, с. 12]
\subsection{Бесконечность множества неподвижных точек в смысле теоремы Клини. Теорема о рекурсии как следствие теоремы Клини. Пример применения теоремы о рекурсии.}

Бесконечность множества неподвижных точек в смысле теоремы Клини: пусть $U$ - г.у.в.ф. и $f\colon\N\to\N$ -- тотальная, тогда бесконечно множество $X$, состоящее $n$ таких, что $U_n=U_{f_n}$.

Доказательство: от противного, пусть $X$ конечно, тогда оно разрешимо и существует вычислимая функция $g$, что ни один ее индекс в $U$ не лежит в $X$. Пусть $m$ -- индекс $g$ в $U$. Рассмотрим $h(x)=\begin{cases}
    m,&\text{если }x\in X \\ f(x),&\text{если }x\not\in X
\end{cases}$. В силу разрешимости $X$, $h$ тотальная вычислимая. По теореме Клини, существует $n$, что $U_n=U_{h(n)}$. Разберем случаи: если $n\in X$, тогда $U_n=U_{h(n)}=U_m=g$, что противоречит определению $g$; если $n\not\in X$, тогда $U_n=U_{h(n)}=U_{f(n)}$, но это значит что $n\in X$, противоречие.

\hfill

Теорема о рекурсии: пусть $V\colon\N^2\to\N$ - вычислимая частичная функция. Тогда существует $n$, что $U_n=V_n$

Доказательство: берем $s\colon\N\to\N$, что $V_n=U_{s(n)}$. По теореме Клини существует $x$, что $U_{s(x)}=U_x$. Подставляем этот $x$: $U_x=U_{s(x)}=V_x$.

\hfill

Пример использования теоремы о рекурсии: существует вычислимая функция $f(x)=\begin{cases}
    1,&x=0 \\ x\cdot f(x-1),&x>0
\end{cases}$

Доказательство: построим $V\colon\N^2\to\N$ следующим образом: $V_k(x)=\begin{cases}
    1,&x=0 \\ x\cdot V_K(x_1),&x>0
\end{cases}$. По теореме о рекурсии находим $n$, что для какой-то г.у.в.ф. $U$ выполняется $U_n=V_n$. Индукция по $x$ показывает что функция $U_n$ удовлетворяет условиям. (TODO: каким условиям? почему вся задача вообще не очевидная?)

% Вычислимость индекса композиции вычислимых функций. Совместная рекурсия: решение «систем уравнений» (с тотальными правыми частями). [R, 53, 56]
\subsection{Вычислимость индекса композиции вычислимых функций. Совместная рекурсия: решение «систем уравнений» (с тотальными правыми частями).}

Вычислимость индекса композиции вычислимых функций: для г.у.в.ф. $U$ существует вычислимая тотальная функция $c\colon\N^2\to\N$, что для любых $p,q\in\N$ выполняется $U_{c(p,q)}=U_p \circ U_q$

Доказательство: возьмем вычислимую $V(n,x)=(U_{\pi^1(n)}\circ U_{\pi^2(n)})(x)$. Существует тотальная вычислимая $s\colon\N\to\N$, что $V_n=U_{s(n)}$. Положим $c(x,y)=s(\langle x,y \rangle)$ и имеем: $U_{c(p,q)}=U_{s(\langle p,q\rangle)}=V_{\langle p,q \rangle}=U_{\pi^1(\langle p,q \rangle)}\circ U_{\pi^2(\langle p,q \rangle)}=U_p \circ U_q$

\hfill

Совместная рекурсия: 

Пусть функции $V_1, V_2: \mathbb{N}^3 \to \mathbb{N}$ вычислимы. Тогда существуют $a, b \in \mathbb{N}$, т.ч. для всех $x \in \mathbb{N}$ выполнены

$$ U(a, x)=V_1(a, b, x) \quad \text{и} \quad U(b, x)=V_2(a, b, x) $$

Доказательство: 

По главности $U$ возьмем $p_1, p_2$ т.ч

$$ U_{p_1} \simeq \pi^1, \quad U_{p_2} \simeq \pi^2. $$

Для вычислимой функции $V: \mathbb{N}^2 \to \mathbb{N}$, т.ч. для всех $k, x \in \mathbb{N} $
$$ V(k, x)= \langle V_1(c(p_1, k), c(p_2, k), x), V_2(c(p_1, k), c(p_2, k), x)\rangle $$

согласно лемме о рекурсии, найдется число $n \in \mathbb{N}$, т.ч. $U_n=V_n$. Положим $a=c(p_1, n)$ и $b=c(p_2, n)$. Тогда для любого $x \in \mathbb{N}$

$$ U(a, x) \simeq U(c(p_1, n), x) \simeq U_{p_1}(U_n(x))\simeq \pi^1(V_n(x)) \simeq \pi^1(V(n, x)) $$

Здесь остановимся и вспомним, что $V$ определена как $\langle \V_1(...), \V_2(...) \rangle$, а $\pi^1$ "расшифровывает" первую координату. Следовательно для любого $x \in \mathbb{N}$

$$\pi^1(V(n, x)) \simeq V_1(c(p_1, n), c(p_2, n), x) \simeq V_1(a, b, x)$$

Для $b$ доказательство аналогичное

\end{document}
