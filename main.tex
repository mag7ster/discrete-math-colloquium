\documentclass[a4paper, 10pt]{article}
\usepackage[unicode=true, colorlinks=true, linkcolor=blue, urlcolor=blue]{hyperref}
\usepackage[T2A]{fontenc}
\usepackage[utf8]{inputenc}
\usepackage[russian]{babel}
%\usepackage{indentfirst}
\usepackage{amssymb}
\usepackage{enumitem}
\usepackage{hyperref}
\usepackage{geometry}
\usepackage{mathtools}
\usepackage{setspace}
\usepackage{tikz}
\usepackage{textcomp}
\usepackage{ stmaryrd }
\usepackage{ulem}
\usepackage{ dsfont }
\usepackage{tikzsymbols}
\usepackage{graphicx}
\usepackage{graphicx}
\graphicspath{{img/}}
\DeclareGraphicsExtensions{.pdf,.png,.jpg}

% Эта штука делает так, что у первого абзаца есть отступ. Почему-то по дефолту в техе все отступы начинаются со второго. Можешь закомментировать и посмотреть, что из этого получится
\usepackage{indentfirst}
\newcommand{\FF}{\mathcal{F}}
\newcommand{\RR}{\mathcal{R}}
\newcommand{\CC}{\mathcal{C}}
\newcommand{\MM}{\mathcal{M}}
\newcommand{\II}{\mathcal{I}}
\newcommand{\NN}{\mathcal{N}}

\geometry{a4paper,top=2cm,bottom=2cm,left=2cm,right=2cm}
\usepackage{hwspecial}

%$$\includegraphics{}$$

\title{Коллоквиум по дискретной математике 2}
\date{}
\begin{document}

\maketitle

\tableofcontents

\newpage

\section{Логика и машины Тьюринга}

\subsection{Структуры и сигнатуры. Нормальные структуры. Изоморфизм структур.}

Структура -- кортеж множеств $(M, \FF, \RR, \CC)$, где 
\begin{enumerate}
    \item $M$ -- непустое множество, \textit{носитель структуры}
    \item $\FF$ -- множество функций вида $f\colon M^n\to M$
    \item $\RR$ -- множество кортежей из $M$
    \item $\CC$ -- подмножество $M$
\end{enumerate}

Сигнатура -- кортеж попарно непересекающихся множеств $(Fnc,Prd,Cnst)$, где $Fnc$ -- множество функциональных символов, $Prd$ -- непустое множество предикатных символов и $Cnst$ -- множество константных символов. (просто набор символов)

*$\sigma$-структура (или интерпретация сигнатуры $\sigma$) -- это формально кортеж $\MM = (M,\FF,\RR,\CC,\II)$, где $\II(Fnc)=\FF,~\II(Prd)=\RR$ и $\II(Cnst)=\CC$. Вводим обозначения: $\II(Fnc)=f^\MM$, $\II(Prd)=R^\MM$ и $\II(Cnst)=c^\MM$. Для задания $\sigma$-структуры достаточно только $M$ и $\II$.

Нормальная структура -- содержащая двувалентный предикатный символ ``=`` $:=~\{(a, a) \in M^2~|~a \in M\}$, где $M$ -- носитель структуры.

% **Инъекция -- отображение такое, что из $f(x_1)=f(x_2)$ следует $x_1=x_2$.

Изоморфизм структур: интепретации $\MM$ и $\NN$ сигнатуры $\sigma$ с носителями $M$ и $N$ соответственно изоморфны если существует биекция $\eta \colon M \to N$ для которой выполняются следующие свойства:
\begin{enumerate}
    \item $\eta(f^\MM(a_1,\ldots,a_n))=f^\NN(\eta(a_1),\ldots,\eta(a_n))$
    \item $(a_1,\ldots,a_n)\in R^\MM \iff (\eta(a_1),\ldots,\eta(a_n))\in R^\NN$
    \item $\eta(c^\MM) = c^\NN$
\end{enumerate}

\end{document}
